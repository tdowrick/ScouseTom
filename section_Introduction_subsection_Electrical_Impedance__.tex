\section{Introduction}

\subsection{Electrical Impedance Tomography}
Electrical Imepdance Tomography (EIT) is a medical imaging technique which reconstructs the internal conductivity of an object from boundary voltage measurements \cite{Metherall1996}. Existing clinical uses of EIT include imaging the lung \cite{Frerichs_2000}, liver \cite{YOU_2009} and breast \cite{Halter}, and a number of EIT systems are now commercially available. Applications of EIT in the brain include imaging in epilepsy \cite{Vongerichten_2016}, \cite{Fabrizi_2006}, acute stroke \cite{Dowrick_2016}, traumatic brain injury \cite{Manwaring2013} and evoked potentials (EPs) \cite{Aristovich_2016}. It is also being developed for imaging fascicle activity in peripheral nerve \cite{KirillStockholm}. 

\subsection{EIT Hardware}

A typical EIT system consists of a current source, voltage measurement unit, electrode array and some form of control/switching circuitry for automating the measurement process. A sinusoidal current of a defined amplitude and frequency is injected through a pair of electrodes, with voltages measured at all electrodes. Typically, a single `measurement' is defined as the demodulated amplitude averaged over a chosen number of sinewave periods. Voltage recordings can be performed either serially, or in parallel on all electrodes. The process is repeated for a number of different pairs of injection electrodes, where the particular combination of injection electrodes is referred to as the protocol. The complete data set consists of _n_ voltage measurements, where _n_ is equal to the number of injection pairs multiplied by the total number of electrodes. This set of measurements makes up a single frame of data and can be used to reconstruct a conductivity profile of the target object. More typically, the difference between two separate frames of data is calculated and used to image the change in conductivity over a certain period.

Existing EIT systems include the KHU \cite{Hun_Wi_2014}, fEITER \cite{McCann_2011}, Dartmouth EIT System \cite{khan}, Xian EIT system \cite{Shi_Xuetao_2005}, Swisstom Pioneer Set (Swisstom AG, Switzerland) and a number of systems previously developed at UCL \cite{Oh2011} \cite{McEwan_2006}. Where available, the key features of each system have been summarised in table (\ref{TableEITSystems}). Where possible, 
figures for noise and output impedance have been taken from the relevant literature for each device. The testing method reported for each device varies, as such some caution should be used when comparing figures for different systems.
