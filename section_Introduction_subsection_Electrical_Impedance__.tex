\section{Introduction}

\subsection{Electrical Impedance Tomography}
Electrical Impedance Tomography (EIT) is a medical imaging technique which reconstructs the internal conductivity of an object from boundary voltage measurements \cite{Metherall1996}. Existing clinical uses of EIT include imaging the lung \cite{Frerichs_2000}, liver \cite{YOU_2009} and breast \cite{Halter}, with commercial EIT systems now available for clinical use. Brain EIT in other active field of research, with development in applications including epilepsy \cite{Vongerichten_2016,Fabrizi_2006}, acute stroke \cite{Dowrick_2016}, traumatic brain injury \cite{Manwaring2013,fu2014use}, evoked potentials (EPs) \cite{Aristovich_2016} and activity in peripheral nerves \cite{KirillStockholm}. 
\subsection{EIT Hardware}

A typical EIT system consists of a current source, a voltage measurement unit, switching circuitry to address multiple electrodes, and a controller to automate the measurement process. A sinusoidal current of a defined amplitude and frequency is injected through a pair of electrodes, with voltages measured at all electrodes. Typically, a single `measurement' is defined as the demodulated amplitude of the voltage at a single electrode, averaged over a chosen number of sinewave periods. These voltage measurements can be obtained either serially, or in parallel on all electrodes. The process is repeated for a number of different pairs of injection electrodes, where the particular sequence is referred to as the `protocol'. The complete data set consists of \emph{n} voltage measurements, equal to the number of injection pairs multiplied by the total number of electrodes. While it is possible for a single frame of data to be used to reconstruct a conductivity profile of the target object, it is more common for to use the difference between two separate frames, producing an image of the change in conductivity.

Existing EIT systems include the KHU \cite{Hun_Wi_2014}, fEITER \cite{McCann_2011}, Dartmouth EIT System \cite{khan}, Xian EIT system \cite{Shi_Xuetao_2005}, Swisstom Pioneer Set (Swisstom AG, Switzerland) and a number of systems previously developed at UCL \cite{Oh2011} \cite{McEwan_2006}. Where available, the key features of each system have been summarised in table (\ref{TableEITSystems}). Values for noise and output impedance have been taken from the relevant literature for each device where available. The testing method reported for each device varies, as such some caution should be used when comparing figures for different systems.
