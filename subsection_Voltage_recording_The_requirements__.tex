\subsection{Voltage recording}
The requirements for voltage recording are parallel data collection, low noise and the ability to save data for offline processing. EEG amplifiers offer an effective off the shelf solution with high performance systems such as the BioSemi ActiveTwo (Biosemi, Netherlands), actiCHamp (Brainproducts GmbH, Germany) and g.tec HIamp (g.tec medical engineering GmbH, Austria) offering 24-bit resolution and a channel count up to 256. Each system offers a PC GUI for saving data to disk, data streaming over TCP/IP and the option to write custom software to interface with the device. 

These specifications come at the expense of maximum bandwidth, which given hardware anti-aliasing filters, is often in the $<$ 20 kHz range table \ref{tableeeg}. This limits the maximum carrier frequency to below approximately 15 kHz, which is sufficient for all brain EIT applications \label{table_requirements}. However, it may preclude use of the system in applications such as lung ventilation, which typically employ frequencies at 50 kHz or above \cite{Frerichs_2000}. For the experimental work presented here, either the BioSemi ActiveTwo (Biosemi, Netherlands) or actiCHamp (Brainproducts GmbH, Germany) system was used for voltage recording. 
