\subsubsection{Multifrequency}

Multifrequency EIT data was collected as part of clinical trial in collaboration with the Hyper Acute Stroke unit (HASU) at University College London Hospital (UCLH). Stroke patients were fitted with 32 commercial EEG electrodes using the same methodology as described in section \ref{methodsTD}. A total of 31 injection pairs were used, with current injected at 17 frequencies between 5 Hz to 2 kHz, for 32 periods at carrier frequencies below 200 Hz, 64 periods between 200 and 1 kHz, and 128 periods above 1 kHz. The amplitude of injected current varied with frequency as per the guidelines in IEC 60601 \cite{IEC}, with the exception of frequencies below 200 Hz which, as as in previous studies \cite{McEwan_2006}, were reduced by half to ensure the current was not perceptible and to avoid saturation of the EEG amplifier. Given the large number of frequencies and the The protocol was repeated three times, taking a total of 20 minutes to complete. 

A further dataset with 60 repeats using only three frequencies, 200 Hz, 1.2 kHz and 2 kHz, was also recorded on each patient, to better evaluate the frequency dependence of the noise and drift performance of the system. Data were collected in 23 patients, for a total of 31 datasets with 17 frequencies and 29 longer term, 3 frequency recordings.[must clarify the length of recordings and data analysis] 