\subsection{Design criteria and technical limitations}
The design criteria set out initially were necessarily broad in order for the system to adapt to the four modes of operation successfully. The configurability of the system meets the requirements for range of injected currents, electrode count, synchronisation and recording of simultaneous EEG signals. In doing so, the system has enabled experiments not previously possible, particularly when used in triggered averaging mode. The noise performance of the system was comparable to existing EIT systems, and was consistent across frequency. Whilst the  $<0.1$ \% noise criteria was met in epicortical recordings, the noise in scalp recordings was closer to $0.65$ \%, which may preclude imaging epileptic seizures using the system in its current form. However, use of a higher frequencies would allow for higher currents to be used, as well as reducing in band physiological noise. Therefore, future experiments should be performed with higher bandwidth systems such as the actiCHamp or g.tec HIamp, with a higher carrier frequency. 

The main limitation of the ScouseTom is the maximum usable frequency of 20kHz imposed by the bandwidth of the EEG systems. This range is sufficient for EIT of fast-neural and stroke, where the impedance contrast is limited to frequencies $<$ 5 10 kHz \cite{Malone2014a,Aristovich_2016,Vongerichten_2016}, but other brain monitoring applications may benefit from the higher current injections afforded at frequencies $>$ 50 kHz \cite{Fabrizi_2006,fu2014use,Manwaring2013}. Currently, the majority of the data processing is not performed in "real time", as with some commercially available systems. However, as most EEG recorders provide open source software, or allow streaming of data via TCP/IP, suitable software can be developed to allow for real time applications where needed. 