\subsubsection{Time difference}
\label{methodsTD}
'Standard' time difference EIT data was collected for imaging stroke in a rat model \cite{Dowrick_2016}. Using 100\( \mu \)A at 2kHz and c. 60 injection pairs, a frame of EIT data was recorded every minute, over 30 minutes. [another backwards sentence...Id redo as ; Recordings made in the anethetised rat... electrodes were, recording was......uL of hep blood were introduced into the at 10 min. This produced an impedance increase of... as the resistance of blood is .... (ref) comapred to that of brain (ref). Must always give sufficient standalone technical detail so reader cna grasp the measurement without the need to look up another paper.] A haemorrhage model where blood was injected into the brain over a 10 minute period, was triggered at the 10 minute mark, which was expected to cause a net impedance decrease. EIT images were produced using a 0th order Tikhonov algorithm on a 4 million element mesh of a generic rat skull [give ref and should introduce the image recon and stats used under exptl design section].

To assess the performance of the system in a clinically realistic scenario, recordings were made in 10 ?healthy subjects [need more details ? any neurological conditions excluded] at rest with 32 commercial EEG electrodes (EasyCap, Germany, www.easycap.de) configured using the standard EEG 10-20 \cite{Jasper1958} and 10-10 extension \cite{Oostenveld2001} systems. To apply electrodes, the skin was cleaned with alcohol and abraded before applying conductive paste [details ? Nu-prep and manuf details]. Current was injected at 1.2 kHz and 160 $\mu$ A, for 64 cycles or 54 ms, between 31 pairs of electrodes, for a total of 930 measurements for a complete frame. This protocol was repeated 60 times over the course of 20 minutes for each patient. In subsequent analysis, channels below 1 mV standing potential were considered, totalling 540. 