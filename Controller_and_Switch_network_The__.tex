\subsection{Controller and Switch network}

The system controller is based upon the Arduino development platform (Arduino LLC), specifically the Arduino Due, and two bespoke PCBs: a controller board or "shield" and a switch network board (highlighted in figure \ref{STOverview}). The controller shield contains the additional circuitry required for isolating the communication with other components of the system, namely RS232 and trigger-link connections with the current source, TTL with the EEG systems, and SPI connection to the switch networks. Due to the modular nature of the design and the variation of use cases, each connection was separately isolated from the mains supply to ensure correct isolation regardless of experimental setup.

The current source is programmed via RS232 by the controller, with external triggering and zero-phase marker signals transmitted via a trigger-link cable. This enables the phase of the injected current to be reset upon switching, or shorter injections to be retriggered by an external pulse. Following the suggestions in \cite{Aristovich_2015}, the variable pulse width stimulation triggers used in EP studies are randomised with respect to the injected current based on the phase marker signals sent by the Keithley 6221.[another backwards sentence - please change to... something like - for recording of evoked cortical responses in humans, a repeated stimulus is employed. {then clarify whether our system sets the pulse or it is external} Timing pulses are then....] Alongside the TTL triggers, a simultaneous stimulation pulse with a programmable range of 3 to 18V can be sent for direct electrical stimulation.

Coded reference pulses from the controller are recorded by the EEG system and uses as reference during data processing. Pulses are sent to indicate the start and stop of current injection, the switching of electrode pairs, changing of injected frequency, stimulation triggers, and the out-of-compliance status of the current source. 

The switch networks comprise two individual series of daisy chained ADG714 CMOS switches (Analog Devices, USA), one each for the source and sink connections from the current source. The switch networks themselves can be daisy chained together, thus enabling current injection between any two electrodes from a total of 128. The switchboard is powered via lithium-ion polymer battery, and communicates with the controller through a digitally isolated SPI bus, figure \ref{STBlocks}. The possible time between switching electrode injection pairs ranges from 100 uS to approximately 70 minutes for 128 channels, which is more than sufficient to meet the frame rate requirements all use cases. 

\subsection{Controller software}

Aside from its ubiquity and open source code, the additional benefit of basing the controller on the Arduino platform is that it is not hardware specific. Thus, porting the software to another device with different architecture is comparatively simple, and does not constrain the system to a specific board in future iterations. Currently, the PC software for serial communication with the controller is written in Matlab (The MathWorks Inc.), but the commands can be easily replicated in another language.

\subsection{Data processing software}

With conventional EIT systems, demodulation of the AC signal is performed in hardware, and only the averaged value for each short injection is transmitted and stored in the PC; the user has little control over the parameters used during processing. Whilst it is possible to alter the firmware for research devices such as the KHU \cite{Hun_Wi_2014} and UCLH \cite{McEwan_2006}, technical constraints such as on-board RAM and limited processing time heavily reduce the versatility of these systems. 

As the ScouseTom system stores the continuous modulated voltages via an EEG system, the demodulation and data processing must be implemented in PC software after the experiment. The processing software is written in Matlab. It employs Zero-phase IIR band pass filtering and the Hilbert transform to produce the envelope and phase of the amplitude modulated signal. The simultaneous EEG signal is obtained simply through low pass filtering the recorded signal. This also allows the user control over all the parameters during demodulation and averaging, demodulation method, centre frequency, bandwidth, filter type etc. This versatility is necessary to process data in EP studies, when the system is used in Triggered mode, as the measurement time is orders of magnitude longer than that of conventional EIT, and the synchronised EEG signal is essential. 

\subsection{Electrode Connectors}

Standard 9 and 37 D-Sub connectors are used for internal connections within the system. Custom connectors and PCBs have been created to allow connection to a range of common electrode interfaces, including EEG electrodes, depth electrodes and Omnetics (Omnetics Connector Corp. MN, USA) devices. Bespoke connectors and cabling can be added to the system for use with non-standard electrode arrays.
