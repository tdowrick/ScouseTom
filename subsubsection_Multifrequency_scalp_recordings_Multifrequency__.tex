\subsubsection{Multifrequency - scalp recordings}

Multifrequency EIT data was collected as part of clinical trial in collaboration with the Hyper Acute Stroke unit (HASU) at University College London Hospital (UCLH). The injection protocol comprised the same 31 pairs used in previous scalp recordings, section \ref{methodsTD}.  Current was injected at 17 frequencies evenly spaced across the usable range of the BioSemi system, \emph{i.e.} from 5 Hz to 2 kHz. As with previous multifrequency experiments, section \ref{MethodsRPFreq}, the length of each injection was frequency dependent to reduce the total recording time. Current was injected for the equivalent of 32 periods at carrier frequencies below 200 Hz, 64 periods between 200 and 1 kHz, and 128 periods above 1 kHz. The amplitude of injected current varied with frequency as per the guidelines in IEC 60601 \cite{IEC}, except frequencies below 200 Hz which, as in previous studies \cite{McEwan_2006}, were reduced by half to ensure the current was not perceptible and to avoid saturation of the EEG amplifier due to the larger contact impedance at these frequencies. The protocol was repeated three times, taking a total of 20 minutes to complete. A further dataset using only three frequencies, 200 Hz, 1.2 kHz and 2 kHz, was also recorded on each patient, to better evaluate the frequency dependence of the noise and drift performance of the system. In this case, a total of 60 frames were collected over the course of 25 minutes. Data were collected in 23 patients, for a total of 31 datasets with 17 frequencies and 29 longer term, 3 frequency recordings. After rejection of negligible channels, each dataset comprised 540 voltages per frequency.  