\subsection{Experimental Design}

First, experiments were carried out on a resistor phantom to characterise the system using similar methods  used to evaluate other systems in the literature, enabling comparisons to be drawn. Subsequently, the system was utilised in all four EIT applications previously described, and the performance evaluated in each case. In some instances, the data were collected during the course of other studies, but evaluation of the performance of the system was not the focus in these cases. Thus further analysis was required to characterise the quality of the data, beyond successfully imaging the impedance changes of interest. 

The initial resistor phantom experiments were conducted to characterise the performance of the system across frequency, and to obtain measures similar to those quoted for other EIT systems. Direct comparisons are made difficult as the exact testing conditions and analyses are not always explicitly stated, e.g. load used, which measurements were included and how many repetitions were used in calculations. Therefore, we have chosen to use phantoms and measurement paradigms which represent realistic usage in brain EIT applications. Specifically, the injected current amplitudes complied with IEC 60601-1 \cite{IEC} safety limits, the number of measurements and the averaging time was chosen to match those used to collect data for real EIT reconstructions. Further, results are presented as a mean and standard deviation across all voltage measurements, with the exception of those with a standing potential below a threshold based on the results from \citet{packham2012comparison}.

hasdasda

Thus the results represent the realistic best case performance of the system in a animal or human experiment, as opposed to the maximum achievable on a test bench. The characteristics of interest were the noise and accuracy over the full frequency range, and the stability of the system over time. 


To assess the system when used in standard time difference mode, data was collected in healthy subjects using scalp EEG electrodes to represent the expected resting noise during scalp EIT recordings \cite{Fabrizi_2006,fabrizi2007analysis,Romsauerova2006}. Additionally, the suitability of the system for imaging impedance changes resulting from stroke in rats was assess from further analysis of data acquired during a feasibility study by \citet{Dowrick_2016}. The system was then utilised in measurements of evoked activity in the rat somatosensory cortex, using an  methodology established using a system developed previously in the UCL group \cite{Oh2011,Aristovich_2016}. This enables a direct comparison of both data and resultant image quality with the system described in this study. Multi-frequency data were collected as part of a larger study in stroke patients in collaboration with University College London Hospital (UCLH). The noise performance of the system was evaluated across a range of frequencies used in a previous simulation study \cite{Malone2014a}, with a subset of frequencies recorded for longer period of time using the same paradigm as the healthy subjects in time difference mode. This data was used to investigate both the variations in performance of the system over the total frequency range, and also the degradation in data quality resulting from transferring from laboratory to clinical measurements. Finally the system was used in the impedance sweep mode of operation in a study by \citet{Dowrick_2015}, characterising the impedance of healthy and ischaemic rat brain \emph{in vivo}. 



