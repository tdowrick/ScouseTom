## Voltage recording
The requirements for voltage recording are parallel data collection, low noise and the ability to save data for offline processing. EEG amplifiers offer an effective off the shelf solution and a number of high performance systems are available[always try to avoid information free phrases like "a number of" - always give some interesting info something like existing systems include......], table \ref{tableeeg}. Each system offers a PC GUI for saving data to disk, data streaming over TCP/IP and the option to write custom software to interface with the device. One drawback of using an EEG amplifier is that the maximum sampling rate limits the EIT injection current to tens of kHz [ backwards confusing sentence - please try to avoid. Change to : The sampling rate and anti-aliasing spec of the EEG amp is... THis limits the EIT carrier frequency to.....] For brain EIT, this is sufficient as the frequencies of interest are typically in the kHz range \cite{Aristovich_2015}, but it may limit the use of the system in certain applications such as Lung ventilation, which typically employ frequencies at about 50 kHz \cite{Frerichs_2000}. For the experimental work presented here, either the BioSemi ActiveTwo (Biosemi, Netherlands) or actiCHamp (Brainproducts GmbH, Germany) system was used for voltage recording [need a sentence somewhere introducing these two systems their spec and manufacturer details].





