\subsection{Purpose}

The motivation behind this work was to develop an EIT system that could be used primarily for imaging in the head and brain, while offering the maximum versatility, such that it could be easily reconfigured for different experimental methods. The following criteria were established:


\begin{itemize}
\item Arbitrary control over current amplitude, up to 10 mA and frequency in the kHz range. The current levels required for EIT can vary from tens of $\mu$A, for epicortical recordings in rats \cite{Oh2011}, to several mA when using human scalp electrodes \cite{tidswell2001three}.
\item Parallel recording of all voltages for processing of data 'offline', allowing for additional signals, such as EEG/ECoG to be recorded alongside the EIT signal.
\item Noise, frame rate and performance characteristics comparable to existing EIT systems. Target resistance changes of $\approx$0.1 \% have been previously identified for EIT recordings of fast neural activity related to neuronal depolarization and scalp recordings of epilepsy \cite{Oh2011,Fabrizi_2006}. This requires stable current injection, <$$0.1\% noise, and voltage recording with accuracy of $\approx$100nV.
\item Variable electrode count, up to a maximum of 256 electrodes.
\item Ability to synchronise injection/recording with external triggers (whisker stimulation, visual, auditory). The phase of injected current should be randomised with respect to the stimulation, to minimise the phase related artefacts in EP recordings \cite{Aristovich_2015}.
\item Reconfigurable modes of operation, to allow for new functionality to be introduced as a later date. Ideally, this should be achievable  through software or firmware changes only.
\item Easily reproducible. Currently, construction of a non-commercial EIT system can take several months and existing publications on EIT systems typically lack enough detail to allow replication. A system which can be easily replicated using a mixture of off the shelf equipment, alongside open source software and hardware designs, will significantly reduce the workload and allow new systems to be assembled in a matter of weeks.
\end{itemize}
