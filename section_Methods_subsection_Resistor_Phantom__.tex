\section{Methods}
\subsection{Resistor Phantom - System Characterisation}

To assess the noise and drift characteristics of the system, experiments were performed on the Cardiff phantom \cite{griffiths1995cole}, configured to a purely resistive load. Voltages were recorded with the BioSemi EEG system, in a configuration used during recordings on the human scalp: with current injected at 2 kHz and 100 $\mu$ A, for 100 ms per measurement, over the course of four hours. The injection electrode protocol was one previously designed for recordings in the head, with 34 different injection pairs and 16 measurement electrodes, for a total of 544 voltage measurements \cite{Fabrizi2009}. This total was reduced to 363 after rejection of measurements on injection electrodes and boundary voltages below 250 $\mu$V. The injection protocol took was repeated every 3.4 seconds over the course of four hours, resulting in a total of 4235 frames. 

As with other EIT systems \cite{oh2007multi} the noise was considered across three time intervals: 100 repeats (approximately five minutes) representing a typical short term measurement, and one and four hours. These longer recordings replicate the use of the system in long term time difference recordings in monitoring applications \cite{fu2014use} \cite{adler2012whither} as well as in triggered mode during EP studies \cite{Aristovich_2016}.

\subsection{Resistor Phantom - Frequency response}

[this whole section unclear - please redo and clarify - what is the phantom - needs clear explanation. Again need to explain the idea behind this and choice of recording - I would do under exptl design. THen here state clearly exactly what you did.  Must explain what you mean by "cycles" "periods" "measurement combinations". ] 

The frequency response was determined using two EEG systems, through measurements on a 32 channel resistive phantom with a current of 100 $\mu$ A. A single measurement [unclear - define ? single el comb - if so which one and how chosen] was averaged across 32 periods [unclear define] for frequencies below 200 Hz and 64 cycles above, with a total of 64 measurement combinations [unclear define] and 10 repeats. Data was collected over a range 20 Hz to 2 kHz, for the BioSemi system and 20 to 20 kHz for the ActiChamp system, and the results normalised to the lowest frequency.  Ideally with a purely resistive test object, no change in voltage should be seen, and the results of these test will be used to calibrate data in all subsequent experiments.