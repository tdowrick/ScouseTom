\section{Methods}


\subsection{Resistor Phantom - System Characterisation}

[need a statement of purpose and exptl design please before methods. Youve a motivation section explaining the design but somewhere need a section introducing testing how]
Resistor phantom measurements were carried out in order to assess the noise and drift characteristics of the system. The 32 channel Cardiff phantom \cite{griffiths1995cole} was used, configured to a purely resistive load.  Data was collected using the BioSemi EEG system, in a configuration similar to that used during recordings on the human scalp: with current injected at 2 kHz and 100 $\mu$ A, for 100 ms per measurement, over the course of four hours.[need to explain the electrode protocol and number of el comb, how long each full data set takes] The noise, defined as the standard deviation, was considered across three time intervals: 100 repeats (approximately five minutes) representing a typical short term measurement, and one and four hours. These longer recordings replicate the use of the system both in triggered mode, as well as long term monitoring applications \cite{fu2014use} \cite{adler2012whither}.[need to explain which channels the analysis was done on and justify. such a choice should go under the exptl design section - here justify how many channels analysis done one and why and also which choice of current and freq with reference to literature. then here in methods just give the techn details.]

\subsection{Resistor Phantom - Frequency response}

[this whole section unclear - please redo and clarify - what is the phantom - needs clear explanation. Again need to explain the idea behind this and choice of recording - I would do under exptl design. THen here state clearly exactly what you did.  Must explain what you mean by "cycles" "periods" "measurement combinations". ] The frequency response was determined using two EEG systems, through measurements on a 32 channel resistive phantom with a current of 100 $\mu$ A. A single measurement [unclear - define ? single el comb - if so which one and how chosen] was averaged across 32 periods [unclear define] for frequencies below 200 Hz and 64 cycles above, with a total of 64 measurement combinations [unclear define] and 10 repeats. Data was collected over a range 20 Hz to 2 kHz, for the BioSemi system and 20 to 20 kHz for the ActiChamp system, and the results normalised to the lowest frequency.  Ideally with a purely resistive test object, no change in voltage should be seen, and the results of these test will be used to calibrate data in all subsequent experiments.