With the exception of those developed at UCL, the systems described were built primarily for EIT imaging in the torso, with little emphasis placed on capturing the data required for EIT imaging of the brain. While the optimal measurement paradigms for EIT of the head and fast-neural activity have not yet been established, the technical requirements for a range of experimental paradigms are known (table \ref{table_requirements}). Given limitations in either hardware, firmware and software it may not be possible to successfully translate existing systems to brain EIT applications. In particular, EIT systems typically perform all demodulation in hardware, transmitting only the amplitude averaged over a chosen number of periods of the carrier frequency. Therefore it is not possible to obtain a continuous impedance signal or record additional signals such as EEG, both of which are necessary for EIT of fast-neural activity \cite{Aristovich_2016}. Further, performing the signal processing in hardware severely limits the ability to adjust parameters such as carrier frequency, measurement speed, and filter bandwidth to meet the different application requirements.