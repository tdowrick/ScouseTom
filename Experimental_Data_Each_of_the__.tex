##Experimental Data

Each of the modes of operation previously described have been used in experimental settings. 

\subsection{Time difference}
During the baseline period before haemorrhage onset, voltages changes of 20.6\(\mu\)V \(\pm\) 51\(\mu\)V (n=7) were observed \ref{FigureStroke}. Over the 10 minute injection period, changes of 20.6mV \(\pm\) 5.4mV occured, with a further increase of 2.2mV \(\pm\) 1.4mV over the remaining measurement time. The time course of the impedance change was consistent with the injection period of the blood into the brain. Previous investigation into data rejection methods for measurements using collected with other EIT systems, suggested removal of up to 30 \% of all channels in phantoms were necessary \cite{packham2012comparison}. However, only 10 \% of channels were rejected from this dataset using similar criteria, suggesting an improvement in the quality of the data, despite the more complex experimental scenario.[unclear and subjective - absolutely cannot introduce this under results which must just be factual. If you'd like to keep this, then introduce under exptl design and justify it and here jsut state no of channels removed. However, even then this eems dodgy - it would be better to say something like - channels removed for these crtieria. Here just say how many met those criteria. Under discussion say better than other studies but need hard evidence not hand-waving to support the point.

\subsection{Scalp Recordings}
The mean signal amplitude across all subjects was 5.4 mV $\pm$ 1.97 mV, and the mean drift across the length of the recordings was 15.1 $\mu$V or 0.26 \%. The mean noise across all 540 measurements and 60 repeats ranged from 8.55 $\mu$V to 53.9 $\mu$V between patients, with an overall noise of 37.2  $\mu$V $\pm$ 39.0 $\mu$V across all measurements, repeats and subjects. The equivalent SNR was 46.5 dB $\pm$ 4.14 dB, ranging from 41.6 to 56.0 dB between subjects. The degradation of the SNR using scalp electrodes is in agreement with previous studies \cite{fabrizi2007analysis} which demonstrated system noise accounts for approximately 5 \%, with physiological noise dominating. As with the long term resistor phantom recordings, the drifts across the recordings reduced the equivalent SNR. Additionally, artefacts resulting from subject or electrode movement were present some recordings, further reducing the effective SNR in these cases.  

\subsection{Triggered}

Despite the 4-fold increase in bandwidth and consequent noise increase, the performance of the ScouseTom system is comparable to that of previous systems \cite{Oh2011} with 0.37 $\mu$V $\pm$ 0.048 $\mu$V noise across all channels. When expressed as a percentage of the baseline voltage, the average noise was 0.009 \% $\pm$ 0.005, thus meeting the criteria outlined in section \ref{motivation}.  Significant impedance changes (dV $> 3\sigma$), a single injection pair example in figure \ref{EPDZ}, ranged from 1.20 $\mu$V to 18.7 $\mu$V (mean 3.97 $\mu$V), equivalent to 3.13 to 33.0 (mean 8.86) SNR. 

EIT reconstructions with this dataset show the onset at 7ms approximately 1 mm below the surface within the primary somatosensory cortex (S1), before expanding to a larger volume until reaching a maximum between 11-12 ms, figure \ref{EPRecon}, then spreading to adjacent areas and finally disappearing at approximately 18 ms. Images obtained previously with a 32 channel system by \citet{Aristovich_2016} with mechanical stimulation of the rat whisker which were found to be in good agreement with established literature, and cross validated with other imaging techniques both in initial onset \cite{armstrong1991thalamo}, and spread of activity \cite{petersen2007functional}. Whilst cross validation was not performed in this study, the literature is equally well established regarding forepaw stimulation, to which these images are in good agreement \cite{peeters2001comparing} \cite{masamoto2007relationship} \cite{lowe2007small}.

\subsection{Multifrequency}

The variation in SNR across frequency, figure \ref{MFSNR}, averaged over all patients, repeats and measurements was not significant, ranging from 44.1 to 45.5 dB.  However, in all of the frequencies below 200 Hz, the SNR was 1dB less than subsequent frequencies. These correspond to those with a lower number of sine wave periods averaged per measurement, and suggest the number of averages should be increased in order to obtain a more uniform response across frequency. 

The SNR of the longer term recordings also did not demonstrate any frequency dependence, with a mean SNR of 41.6 $\pm$ 7.9 dB, 41.4 $\pm$ 8.25 dB and 40.7 $\pm$ 8.2 dB for 0.2 1.2 and 2 kHz respectively. Similarly the mean drift in voltage was 0.74 \%, 0.61 \% 0.68 \%, and was thus largely unchanged across frequency. The decrease in average SNR compared to the recordings in subjects in section \ref{scalp} is largely a result of a greater prevalence of motion artefacts from patient movement. 


\subsection{Impedance Spectrum Measurement}

Healthy brain tissue showed a non linear decrease of 15\% impedance over 0-250Hz, with ischaemic brain showing a decrease of 7 \% over the same range, with a more linear slope \ref{FigureSweep}. Above 250Hz, the impedance of both tissue types decreased at the same rate. These results were consistent with previous studies in the area \cite{Ranck_1963}, \cite{Logothetis_2007}.