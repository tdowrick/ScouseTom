\subsection{Time difference}
The decrease in SNR in animal and human experiments is in agreement with previous studies \cite{fabrizi2007analysis} which demonstrated system noise accounts for approximately 5 \%, with physiological noise dominating. As with the long term resistor phantom recordings, the drifts across the recordings reduced the equivalent SNR. This physiological variability is evident in the range of SNR between subjects in both rat and human subjects. The system noise is thus masked by electrochemical or physiological changes over the course of the experiment. Despite this, conductivity changes resulting from the haemorrhage model could be imaged and correctly localised in the majority of cases. The SNR in the scalp recordings was decreased further in some recordings by artefacts resulting from subject or electrode movement. Between patients the SNR ranged from 41.6 to 56.0 dB, equivalent to 0.18 \% to 0.94 \%, which was less than the 2 \% observed by \citet{Romsauerova2006} using a similar current amplitude and frequency, and within the 37.9 to 62.8 dB range observed by \citet{Xu2011} using an order of magnitude greater amplitude at 50 kHz. Thus the ScouseTom system offers comparable or greater performance to existing systems, but with increased versatility in experimental settings. 