\subsubsection{Impedance Spectrum Measurement - ischaemic rat brain}

Impedance spectrum measurements were made with the ScouseTom system \cite{Dowrick_2015}, where low frequency ($<$ 3kHz) impedance measurements were made on healthy (n=112 voltage measurements in 4 rats) and ischaemic (n=56 in 2 rats) rat brain. Using an 30 contact epicortical electrode array, section \ref{MTrig}, impedance data was collected using an injected current of 100$\mu$A through a single pair of electrodes, located on opposite corners of the array. The frequency of injected current was increased in 5Hz intervals from 1Hz to 100Hz, 10Hz intervals from 100Hz to 1000Hz and 50Hz intervals between 1000Hz and 3000Hz, for a total of 136, with a minimum of 50 periods of the waveform recorded at each frequency. This sweep was repeated in ascending and descending order and finally with random ordering. The use of an EEG recorder allowed parallel voltage recording at all electrodes, which were averaged together at each frequency. The relative change in impedance, rather than the absolute value, was calculated, by comparing the measured voltages at each frequency.