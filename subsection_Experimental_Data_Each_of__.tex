\subsection{Experimental Data}

Each of the modes of operation previously described have been used in experimental settings. 

\subsubsection{Time difference}
The noise in the baseline recordings (n=115 in N=15) before intervention was 0.64 $\mu$V $\pm$ 2.12 $\mu$V equivalent to 1.90 \% $\pm$ 3.6 \% or 42.4 $\pm$ 11.9 dB. The mean SNR per experiment (N=15) ranged from 28.4 to 62.1 dB. Over the 10 minute injection period, changes of 20.6mV \(\pm\) 5.4mV occurred (n=7) figure \ref{FigureStroke}, with a further increase of 2.2mV \(\pm\) 1.4mV over the remaining measurement time. The time course of the impedance change was consistent with the injection period of the blood into the brain. Physiologically representative localised impedance changes could successfully be reconstructed in 5 out of the 7 experiments, figure \ref{FigureStroke} b. 


\subsubsection{Scalp Recordings}
The mean signal amplitude across all subjects was 5.4 mV $\pm$ 1.97 mV, and the mean drift across the length of the recordings was 15.1 $\mu$V or 0.26 \%. The mean noise across all 540 measurements and 60 repeats ranged from 8.55 $\mu$V to 53.9 $\mu$V between patients, with an overall noise of 37.2  $\mu$V $\pm$ 39.0 $\mu$V across all measurements, repeats and subjects. The equivalent SNR was 46.5 dB $\pm$ 4.14 dB, ranging from 41.6 to 56.0 dB between subjects. The degradation of the SNR using scalp electrodes is in agreement with previous studies \cite{fabrizi2007analysis} which demonstrated system noise accounts for approximately 5 \%, with physiological noise dominating. As with the long term resistor phantom recordings, the drifts across the recordings reduced the equivalent SNR. Additionally, artefacts resulting from subject or electrode movement were present some recordings, further reducing the effective SNR in these cases.  


