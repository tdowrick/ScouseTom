\subsubsection{Triggered Averaging - rat somatosensory cortex}

The ScouseTom system was utilised in measurements of evoked activity using epicortical electrode arrays in anesthetised rats, repeating methodology from previous studies \cite{Aristovich_2016} \cite{Vongerichten_2016}. Arrays of platinised stainless-steel electrodes embedded in silicon, with 0.6 mm diameter contacts spaced 1.2 mm apart were placed directly onto the surface of the brain, targetting the somatosensory cortex. Whilst previous experiments had employed a single grid of 30 electrode contacts on a single side of the brain, in this study two grids of 57 electrodes were used, one on each hemisphere, for a total of 114. The activity was induced via electrical stimuli to the forepaw, delivered in 10 mA pulses of 1 ms duration at a rate of 2 Hz, triggered by the ScouseTom controller.

Current was injected at 1.7 kHz with 50 $\mu$ A amplitude, for 30 seconds across a pair of electrodes, with voltages on 127 electrodes recorded in parallel using the ActiChamp ActiveTwo system with 25 kHz sampling rate. The voltages were demodulated with 1 kHz bandwidth, yielding 2 ms time resolution, and coherent averaging was performed on 60 500 ms trials centered around the time of stimulation. This process was repeated for c. 50 pairs of injection electrodes over the course of 25 minutes to produce a complete data set of c. 6000 voltages. EIT images were reconstructed at each 2ms time point in c. 80000 element hexahedral mesh, using the methodology outlined in \cite{Aristovich_2014}.
