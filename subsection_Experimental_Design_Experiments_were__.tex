\subsection{Experimental Design}

Experiments were initially carried out on a resistor phantom to characterise the system across a range of frequencies. Subsequently, the system was utilised in all four EIT applications previously described, and the performance evaluated in each case. In some instances, the data were collected as part of other studies, Where the main aim was successfully imaging the impedance change of interest. In these cases additional analysis of the data was required to extract the relevant performance characteristics.

For the initial resistor phantom measurements, direct comparisons with systems reported in the literature are difficult to perform as the exact testing conditions and analyses are not always explicitly stated.
Therefore, phantoms and measurement paradigms which represent realistic usage in brain EIT applications were chosen. Specifically, the injected current complied with IEC 60601-1 \cite{IEC} safety limits, the number of measurements and the averaging time was chosen to match those of real experiments. All voltages, rather the subset with highest SNR, were considered in subsequent analysis, with the exception of those with sufficiently low amplitude which are routinely neglected during reconstructions \cite{packham2012comparison}. Thus the results represent the realistic best case performance of the system in a animal or human experiment, as opposed to the maximum achievable on a test bench. The characteristics of interest in these experiments were the noise and drift in measurements over time, and the reciprocity error (RE), common metric for accuracy of EIT systems \cite{Hun_Wi_2014}. 

With a purely resistive test object, the amplitude of the measured voltages should be frequency independent. In reality, the components any system will exhibit some variations with frequency, necessitating some degree of calibration in post processing \cite{Hun_Wi_2014,McEwan_2006}. In the case of the ScouseTom system, the component with the greatest frequency dependence is the anti-aliasing filter within the EEG systems. To study the frequency dependence of the system as whole, changes in amplitude across frequency using two commercial EEG systems were compared to the theoretical changes resulting from their hardware filters. 

Data was collected in healthy subjects using scalp EEG electrodes to represent the expected resting noise during clinical scalp EIT recordings \cite{Fabrizi_2006,fabrizi2007analysis,Romsauerova2006}. Additionally, the suitability of the system for imaging impedance changes resulting from stroke during a feasibility study by \citet{Dowrick_2016} was assessed. The system was then utilised in measurements of evoked activity in the rat somatosensory cortex, using an established methodology \cite{Oh2011,Aristovich_2016}. Multi-frequency data were collected as part of a larger study in stroke patients in collaboration with University College London Hospital (UCLH). The noise performance of the system was evaluated across a frequency range previously used in a simulation study \cite{Malone2014a}, with a subset of frequencies recorded for longer period of time using to assess degradation in data quality a result of transferring from laboratory to clinical measurements. This data was used to investigate both the variations in performance of the system over the total frequency range. Finally, the system was used in the impedance sweep mode of operation in a study by \citet{Dowrick_2015}, characterising the impedance of healthy and ischaemic rat brain \emph{in vivo}. 

