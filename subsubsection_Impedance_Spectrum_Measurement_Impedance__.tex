\subsubsection{Impedance Spectrum Measurement - ischaemic rat brain}

Impedance spectrum measurements were made with the ScouseTom system \cite{Dowrick_2015}, where low frequency ($<$ 3kHz) impedance measurements were made on healthy (n=112 voltage measurements in 4 rats) and ischaemic (n=56 in 2 rats) rat brain. Impedance data was collected using a 100uA injection through two channels of a 30 electrode array, placed directly on the brain surface. The frequency was increased in 5Hz intervals from 1Hz to 100Hz, 10Hz intervals from 100Hz to 1000Hz and 50Hz intervals between 1000Hz and 3000Hz, with a minimum of 50 periods of the waveform recorded at each frequency. The frequency was swept from 1Hz-3000Hz, 3000Hz-1Hz and with random ordering. The use of an EEG recorder allowed parallel voltage recording at all electrodes, which were averaged together at each frequency. The relative change in impedance, rather than the absolute value, was calculated, by comparing the measured voltages at each frequency.