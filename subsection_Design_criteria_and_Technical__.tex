\subsection{Design criteria and Technical Limitations}
The design criteria set out initially were necessarily broad in order for the system to adapt to the four modes of operation successfully. The reconfigurability of the system meets these requirements in terms of range of injected currents, electrode count, synchronisation and recording of simultaneous EEG signals. 

The system meets with all the criteria described as a reconfigurable system adaptable to a variety of fast neural and brain EIT applications, whilst allowing for extensive control over the measurement setup and data processing. The utilisation of system in four main modes of operation has been outlined, facilitating EIT experiments into time imaging of stroke, multi-frequency stroke type classification, stimulation-triggered evoked potentials. 

The use of EEG amplifiers for voltage recording limits the maximum frequency to 20kHz, which is sufficient for fast neural EIT, but may not be sufficient for other EIT applications. Additionally, the BioSemi and actiCHamp systems implement hardware antialiasing filters, which causes a reduced gain at higher frequencies, figure \ref{freqresp}. Given that the frequency response of the gain is known, it can easily be accounted for when using multi-frequency data.

Currently, the majority of the data processing is not performed in "real time", as a complete dataset must be saved to disk. However, as most EEG systems allow for streaming of data via TCP/IP, much of this analysis could be converted to process a single measurement at a time, thus enabling much of the capabilities of other commercial EIT systems.

This calibration has demonstrated substantial benefits with other EIT systems \cite{Hun_Wi_2014} \cite{khan}, but comes at the expense of adaptability.