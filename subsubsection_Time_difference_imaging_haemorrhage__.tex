\subsubsection{Time difference - imaging haemorrhage}
\label{methodsTD}
'Standard' time difference EIT data were recorded in an anesthetised rat, from 40 spring-loaded gold plated electrodes placed on the skull in an even distribution. A model of intracerebral haemorrhage was used wherein 50 $\mu$l of autologous blood was injected via a cannula into the brain at a rate of 5 $\mu$l per minute \cite{Dowrick_2016}. Given the resistance of blood and grey matter at 1 Khz is approximately 1.4 $\Omega {\text{m}}^{-1}$ and 10 $\Omega {\text{m}}^{-1}$ respectively \cite{Gabriel_2009}, this produced up to a seven-fold increase in conductivity localised around the injection site. A current of 100\( \mu \)A at 2kHz was injected, using a protocol of c. 60 injection pairs, with each injection lasting 1 s. A complete frame of EIT data was recorded every minute, over a total of 30 minutes, including 10 minutes prior to the injection of blood.  





A haemorrhage model where blood was injected into the brain over a 10 minute period, was triggered at the 10 minute mark, which was expected to cause a net impedance decrease. EIT images were produced using a 0th order Tikhonov algorithm on a 4 million element mesh of a generic rat skull [give ref and should introduce the image recon and stats used under exptl design section].

\subsubsection{Time difference - noise on scalp}

To assess the performance of the system in a clinically realistic scenario, recordings were made in 10 ?healthy subjects [need more details ? any neurological conditions excluded] at rest 



Current was injected at 1.2 kHz and 160 $\mu$ A, for 64 cycles or 54 ms, between 31 pairs of electrodes, for a total of 930 measurements for a complete frame. This protocol was repeated 60 times over the course of 20 minutes for each patient. In subsequent analysis, channels below 1 mV standing potential were considered, totalling 540. 