\subsubsection{Time difference - imaging haemorrhage}
\label{methodsTD}
'Standard' time difference EIT data were recorded in an anesthetised rat, from 40 spring-loaded gold plated electrodes placed on the skull in an even distribution. A model of intracerebral haemorrhage was used wherein 50 $\mu$l of autologous blood was injected via a cannula into the brain at a rate of 5 $\mu$l per minute \cite{Dowrick_2016}. Given the resistance of blood and grey matter at 1 Khz is approximately 1.4 $\Omega {\text{m}}^{-1}$ and 10 $\Omega {\text{m}}^{-1}$ respectively \cite{Gabriel_2009}, this produced up to a seven-fold increase in conductivity localised around the injection site. A current of 100\( \mu \)A at 2kHz was injected, using a protocol of c. 60 injection pairs, with each injection lasting 1 s. A complete frame of EIT data was recorded every minute, over a total of 30 minutes, including 10 minutes prior to the injection of blood. This experiments was repeated in 7 rats, with time difference images reconstructed every frame with respect to the baseline.

\subsubsection{Time difference - scalp recordings}

To assess the performance of the system in a clinically realistic scenario, EIT recordings were made in 10 healthy volunteers reclining in a comfortable position. Current of 1.2 kHz and 160 $\mu$A was injected for 54 ms, equivalent to 64 sinewave periods, between 31 pairs of electrodes, resulting in a complete frame every 1.6 seconds. The injection pairs were chosen to maximise the number of independent measurements and \cite{Malone2014a}


for a total of 930 measurements for a complete frame. This protocol was repeated 60 times over the course of 20 minutes for each subject. In subsequent analysis, channels below 1 mV standing potential were considered, totalling 540. 