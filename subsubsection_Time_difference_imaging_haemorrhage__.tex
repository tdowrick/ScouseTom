\subsubsection{Time difference - imaging haemorrhage}
'Standard' time difference EIT data were recorded in an anesthetised rat, from 40 spring-loaded gold plated electrodes placed on the skull in an even distribution. A model of intracerebral haemorrhage was used wherein 50 $\mu$l of autologous blood was injected via a cannula into the brain at a rate of 5 $\mu$l per minute \cite{Dowrick_2016}. Given the resistance of blood and grey matter at 1 Khz is approximately 1.4 $\Omega {\text{m}}^{-1}$ and 10 $\Omega {\text{m}}^{-1}$ respectively \cite{Gabriel_2009}, this produced up to a seven-fold increase in conductivity localised around the injection site. A current of 100\( \mu \)A at 2kHz was injected, using a protocol of c. 60 injection pairs, with each injection lasting 1 s. The specific protocol was adapted for each individual experiment, dependent upon the quality of electrode contacts. A complete frame of EIT data was recorded every minute, over a total of 30 minutes, including 10 minutes prior to the injection of blood.  Time difference images were reconstructed with respect to the baseline, for every frame recorded after starting the injection of blood. This procedure was repeated in a total of 7 rats. 

\subsubsection{Time difference - scalp recordings}
\label{methodsTD}
To assess the performance of the system in a clinically realistic scenario, EIT recordings were made in 10 healthy volunteers in a seated position. Current of 1.2 kHz and 160 $\mu$A was injected for 54 ms, equivalent to 64 sinewave periods, between 31 pairs of electrodes, resulting in a complete frame every 1.6 seconds. The injection pairs were chosen to maximise both the number of independent measurements and the overall magnitude of the voltages \cite{Malone2014a}. A complete frame consisted of a total of 930 measurements, 540 of which were considered in subsequent analysis, after rejecting measurements below 1 mV. A total of 60 frames were recorded over the course of 20 minutes for each subject. 