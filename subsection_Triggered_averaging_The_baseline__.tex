\subsection{Triggered averaging}
The baseline noise in these recordings agreed with that observed in previous fast-neural activity experiments, which reported noise of 0.18 $\mu$V $\pm$ 0.04 $\mu$V, using a bandwidth of 250 Hz \cite{Oh2011,Packham2016} and averaging over 120 repeated stimuli. However, the results presented in these studies were obtained after averaging of only 60 trials, suggesting a decrease in background noise when using the ScouseTom system. This not only offers benefits in terms of signal quality, but shorter data collection times reduce the effects of physiological drifts and allow for longer injection protocols, potentially improving EIT reconstructions.  

Previous experiments with this methodology with mechanical stimulation of the rat whisker by \citet{Aristovich_2016} were found to be in good agreement with established literature, and cross validated with other imaging techniques both in initial onset \cite{armstrong1991thalamo}, and spread of activity \cite{petersen2007functional}. Whilst cross validation was not performed in this study, the literature is equally well established regarding forepaw stimulation, and the location and depth of the area of onset in figure \ref{EPRecon} matches expectations \cite{peeters2001comparing} \cite{masamoto2007relationship} \cite{lowe2007small}. Existing 32 channel systems limited measurements to a single hemisphere of the brain, the increased electrode count of the ScouseTom enables investigation into deeper brain structures. 
