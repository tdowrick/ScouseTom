\subsection{EIT applications}
Four main data collection strategies were identified for EIT experiments fig \ref{Modes}.  The most common method of data collection in EIT is time difference, where multiple data sets are obtained at different points in time, \emph{i.e.} before and after the introduction of some perturbation, and the change in impedance over time is reconstructed. Such is the prevalence of this method, it can be considered the 'standard' mode of operation, being used in the vast majority of successful experiments in the field \cite{bayford2012bioimpedance}.

The second method is time difference imaging, using an external triggered stimulus to allow for coherent averaging. Data collection is synchronised to an external stimulus (e.g. whisker stimulator in rats, or visual EPs in humans) which is also recorded. This method, first developed by \citet{Oh2011} is used when impedance signals have low signal to noise ratio (SNR) and millisecond duration. Current is continuously injected, while the stimulus is repeated, generating multiple evoked potentials at different time points. Through coherent averaging, an impedance signal is obtained at each sample interval, yielding millisecond resolution. 

3) Multi-frequency (simultaneous). Multi frequency EIT data can be collected at any combination of frequencies over the operating range. The ordering of frequencies can be fixed, or randomised and the injection period of each particular frequency can be specified. Composite waveforms, consisting of multiple frequencies such as those implemented in the existing KHU \cite{Hun_Wi_2014} and UCL \cite{McEwan_2006} systems, can also be generated. As opposed to Time-difference EIT where the change in conductivity between time points is reconstructed, frequency-difference EIT reconstructs the conductivity changes over frequency. These methods are substantially less robust to modelling and instrumentation errors than standard EIT methods, and thus place more stringent requirements on the algorithms and hardware \cite{Ahn2011} \cite{Malone2014}.

4) Multi-frequency sweep. A frequency sweep can be carried out, where the injection frequency is incremented and measurements taken at each point. Both increasing and decreasing sweeps are possible, as is randomised ordering. This has applications for impedance characterisation, in place of a stand alone impedance analyser \cite{Gabriel_2009}, and selecting an optimal measurement frequency \cite{VongerichtenASantosGAristovichK2013}.