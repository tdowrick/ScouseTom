
\subsection{Controller software}

As the Arduino platform is not hardware specific, porting the software to another device with different architecture is comparatively simple, and does not constrain the system to a specific board in future iterations. Currently, the PC software for serial communication with the controller is written in MATLAB (The MathWorks Inc.) but the commands can be easily replicated in another language.

\subsection{Data processing software}

With conventional EIT systems, demodulation of the AC signal is performed in hardware, and only the averaged value for each short injection is transmitted and stored in the PC; the user has little control over the parameters used during processing. Whilst it is possible to alter the firmware for research devices such as the KHU \cite{Hun_Wi_2014} and UCLH \cite{McEwan_2006}, technical constraints such as on-board RAM and limited processing time impact the versatility of these systems. 

As the ScouseTom system stores the continuous modulated voltages captured via an EEG system, the demodulation and data processing must be implemented in PC software after data collection. Currently, the processing software is written in Matlab. It employs Zero-phase IIR band pass filtering and the Hilbert transform to produce the envelope and phase of the amplitude modulated signal. The simultaneous EEG signal is obtained by low pass filtering the original signal. This also allows the user control over all the parameters during demodulation and averaging, demodulation method, centre frequency, bandwidth, filter type etc. This versatility is necessary to process data in EP studies, when the system is used in Triggered mode, as the measurement time is orders of magnitude longer than that of conventional EIT, and the synchronised EEG signal is essential. 

\subsection{Electrode connectors}

Standard 9 and 37 D-Sub connectors are used for internal connections within the system. Custom connectors and PCBs were created to allow connection to a range of common electrode interfaces, including EEG electrodes, depth electrodes and Omnetics (Omnetics Connector Corp. MN, USA) devices. Bespoke connectors and cabling can be added to the system for use with non-standard electrode arrays.