\section{Discussion}
A new EIT system has been described which can be utilised in a range of experiments, particularly those involving EIT of the head or nerve. A key advantage of the ScouseTom system is the high level of control it offers over all aspects of the experimental process. This versatility is as a result of the stable performance of the hardware across frequency and load, in particular the current source, which largely removes the need for protocol specific calibration. This calibration has demonstrated substantial benefits with other EIT systems \cite{Hun_Wi_2014} \cite{khan}, but comes at the expense of adaptability.

The system was able to detect voltage changes of the order of 10s of $\mu$V in both stand EIT time difference and stimulation triggered experiments. Physiological variations and noise, as well as differing hardware setups and data analysis techniques make direct comparison to other EIT systems difficult. However, the results from previous EP studies \cite{Oh2011} in the rat cortex have demonstrated a similar noise level to those recorded with the system described here, suggesting physiological noise is dominant. This is also true for the scalp recordings, where the system noise was masked by physiological noise as well as motion artefacts, although not to the same extent as had been found in previous studies \cite{Fabrizi_2006}. 
