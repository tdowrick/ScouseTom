\subsection{System characterisation}
Results from the resistor phantom with realistic loads demonstrated a short term SNR of 77.5 dB $\pm$ 1.3 dB, with a reduction of 5 dB over an hour, and a total of 12.6 dB over the whole four hour period. As has  been observed in existing EIT systems \cite{oh2007multi}, this apparent decrease is as a result of low frequency drifts, likely from temperature changes in the current source, rather than degradation of the performance of the system figure \ref{SNR}. Overall, the SNR was lower than figures reported for other systems, which are capable of measuring with SNR above 90 dB \cite{khan,Hun_Wi_2014}. The amplitude of the voltages recorded was 2.66 mV, orders of magnitude less than the input range of the EEG amplifier used ($\approx$ 250mV), and did not take full advantage of the high dynamic range offered by these amplifiers. The other EIT systems in the literature incorporate a programmable gain amplifier before digitisation to overcome this problem, often necessitating detailed calibration before data collection. It is therefore possible that certain measurements with larger injected currents and voltage amplitudes could far exceed the SNR reported in this study. One of the few systems in the literature which targeted the same frequency range, and thus with the same constraints on current and voltage amplitude, demonstrated a noise of 0.1\%, close to an order of magnitude greater than those reported here \cite{McEwan_2006}. The reciprocity error (RE) of 0.42 \% is comparable to other EIT systems, despite the lack of calibration \cite{oh2007multi,Hun_Wi_2014,khan}. This is likely as a result of the comparatively low frequency used in this study, where the effect of stray capacitance in minimal.  

The frequency response of the system as a whole matched that of the EEG systems used, and the SNR was consistent across frequency. This suggests that beyond correcting for the EEG amplifier gain, frequency specific calibration will not offer significant benefits to the accuracy of the system. If higher frequencies ($>20$ kHz) were recorded using a different EEG system, then stray capacitance could no longer be ignored, and calibration would likely be required. 
