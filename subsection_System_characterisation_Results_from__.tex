\subsection{System characterisation}
Results from the resistor phantom with realistic loads demonstrated a short term SNR of 77.5 dB $\pm$ 1.3 dB, with a reduction of 5 dB over an hour, and a total of 12.6 dB over the whole four hour period. As has  been observed in existing EIT systems \cite{oh2007multi}, this apparent decrease is as a result of low frequency drifts, likely from temperature changes in the current source, rather than degradation of the performance of the system figure \ref{SNR}. 

Overall, the SNR was lower than figures reported for other systems, which are capable of measuring with SNR above 90 dB \cite{khan,Hun_Wi_2014}. The amplitude of the voltages recorded was 2.66 mV, orders of magnitude less than the input range of the EEG amplifier used ($\approx$ 250mV), and thus is it not possible to take full advantage of the high dynamic range offered by these amplifiers. The 

The reciprocity error (RE) of 0.42 \% is comparable to other EIT systems, despite the lack of calibration \cite{oh2007multi,Hun_Wi_2014,khan}. This is likely as a result of the comparatively low frequency used in this study, where the effect of stray capacitance in minimal.  








The system was able to detect voltage changes of the order of 10s of $\mu$V in both stand EIT time difference and stimulation triggered experiments. Physiological variations and noise, as well as differing hardware setups and data analysis techniques make direct comparison to other EIT systems difficult. However, the results from previous EP studies \cite{Oh2011} in the rat cortex have demonstrated a similar noise level to those recorded with the system described here, suggesting physiological noise is dominant. This is also true for the scalp recordings, where the system noise was masked by physiological noise as well as motion artefacts, although not to the same extent as had been found in previous studies \cite{Fabrizi_2006}. 

