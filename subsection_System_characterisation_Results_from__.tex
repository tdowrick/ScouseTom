\subsection{System characterisation}
Results from the resistor phantom with realistic loads demonstrated a short term SNR of 77.5 dB $\pm$ 1.3 dB, with a reduction of 5 dB over an hour, and a total of 12.6 dB over the whole four hour period. This has been observed in existing EIT systems \cite{oh2007multi}, where is was attributed to long terms drifts arising predominantly from temperature changes within the current source, and variations in power supply voltage. The stability of the short term SNR in figure \ref{SNR},  



expanded comparison to other systems. we expect lower SNR based on testing paradigm

cardiff phantom appears noisier

The reciprocity error (RE) of 0.42 \% is comparable to other EIT systems, despite the lack of calibration \cite{oh2007multi,Hun_Wi_2014,khan}. This is likely as a result of the comparatively low frequency used in this study, where the effect of stray capacitance in minimal.  








The system was able to detect voltage changes of the order of 10s of $\mu$V in both stand EIT time difference and stimulation triggered experiments. Physiological variations and noise, as well as differing hardware setups and data analysis techniques make direct comparison to other EIT systems difficult. However, the results from previous EP studies \cite{Oh2011} in the rat cortex have demonstrated a similar noise level to those recorded with the system described here, suggesting physiological noise is dominant. This is also true for the scalp recordings, where the system noise was masked by physiological noise as well as motion artefacts, although not to the same extent as had been found in previous studies \cite{Fabrizi_2006}. 

