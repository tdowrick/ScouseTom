\subsection{Experimental data}

To evaluate the system experimentally and clinically, experiments were performed using each the four modes of operation. Certain methods were common across some experiments and are detailed here separately. 

\subsubsection{Scalp electrode preparation}
All recordings on the human scalp were performed using 32 commercial EEG electrodes (EasyCap, Germany), using the configuration described by \citet{tidswell2001three}, which includes 21 locations from the EEG 10-20 standard \cite{Jasper1958} and 11 additional electrodes. For these experiments, the locations were updated to match the nearest equivalents in either the 10-10 or 10-5 extensions \cite{Oostenveld2001}. Each electrode site was first cleaned with surgical spirit, then abraded using Nuprep gel (Weaver and Co., USA), with the electrode finally affixes using Elefix conductive paste (Nihon Kohden, Japan). 

\subsubsection{EIT reconstruction}
When possible, EIT images were reconstructed using the same methodology used in other studies by the UCL group \cite{Dowrick_2016,Aristovich_2016,Aristovich_2014}. First the `forward problem' and sensitivity matrix was calculated using the Parallel EIT Solver \cite{Jehl2014}, in a c. 4 millions element tetrahedral mesh. The linear inverse solution was obtained using zeroth-order Tikhonov regularisation, with the hyperparameter $\lambda$ chosen through cross-validation.  The conductivity changes were reconstructed in a lower resolution hexahedral mesh of approximately 100,000 elements, and we were subsequently post-processed using a noise based correction to produce images of significance values \cite{Aristovich_2016}.