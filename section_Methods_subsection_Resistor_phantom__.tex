\section{Methods}
\subsection{Resistor phantom - system Characterisation}

To assess the noise and drift characteristics of the system, experiments were performed on the Cardiff phantom \cite{griffiths1995cole}, configured to a purely resistive load. Voltages were recorded with the BioSemi EEG system, in a configuration used during recordings on the human scalp: with current injected at 2 kHz and 100 $\mu$A, for 100 ms per measurement, over the course of four hours. The injection electrode protocol was one previously designed for recordings in the head, with 34 different injection pairs and 16 measurement electrodes, for a total of 544 voltage measurements \cite{Fabrizi2009}. This total was reduced to 363 after rejection of measurements on injection electrodes and boundary voltages below 250 $\mu$V. The injection protocol took was repeated every 3.4 seconds over the course of four hours, resulting in a total of 4235 frames. 

As with other EIT systems \cite{oh2007multi} the noise was considered across three time intervals: 100 frames (approximately five minutes) representing a typical short term measurement, and one and four hours. These longer recordings replicate the use of the system in long term time difference recordings in monitoring applications \cite{fu2014use} \cite{adler2012whither} as well as in triggered mode during EP studies \cite{Aristovich_2016}. To evaluate the drift in performance over time, the change in voltages over the four hour recording was also calculated, as well as the SNR for each block of 100 frames. The reciprocity error (RE) was calculated using the same 100 frames as the noise analysis, and expressed as an absolute error in percentage.

\subsection{Resistor phantom - frequency response}
\label{MethodsRPFreq}
The same resistor phantom was used to investigate the frequency response of the system, using two EEG systems. Data was collected at 15 frequencies over a range 20 Hz to 2 kHz using the BioSemi ActiveTwo and 33 frequencies between 20 Hz to 20 kHz for the actiCHamp system. To reduce the data collection time, six injection pairs from the protocol used previously were chosen, resulting in a total of 64 voltage measurements for analysis. The time per measurement was frequency dependent, equivalent to 32 sine-wave periods for frequencies below 200 Hz and 64 periods for those above. The whole protocol was repeated for 10 frames, with the frequency order randomised within each frame. The current amplitude was 100 $\mu$A for all frequencies. The voltages recorded were normalised with respect to the amplitude at 20 Hz, and expressed as a percentage. These results was compared to the expected decrease resulting from the hardware filters in each system. To asses the noise frequency dependence of the noise performance, the SNR across all frequencies was also calculated.  


