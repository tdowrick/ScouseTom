\subsection{System characterisation}
Experiments in a resistor phantom with realistic loads demonstrated an SNR of 77.5 dB $\pm$ 1.3 dB 




when averaged across 100 repeats of a 540 measurement protocol. There were no significant changes in SNR across a four hour recording, or across the frequency range of both EEG systems. 


expanded comparison to other systems. we expect lower SNR based on testing paradigm

cardiff phantom appears noisier
These arise predominantly from temperature changes within the current source, and variations in power supply voltage.



RE as good as KHU and others


\subsection{Time difference}


 The degradation of the SNR using scalp electrodes is in agreement with previous studies \cite{fabrizi2007analysis} which demonstrated system noise accounts for approximately 5 \%, with physiological noise dominating. As with the long term resistor phantom recordings, the drifts across the recordings reduced the equivalent SNR. Additionally, artefacts resulting from subject or electrode movement were present some recordings, further reducing the effective SNR in these cases.  
 
 and noise drifts increase from resistor phantom to human - physilogical noise dominates
 

\subsection{Triggered averaging}


noise in trig same as others - system not limit



Despite the 4-fold increase in bandwidth and consequent noise increase, the performance of the ScouseTom system is comparable to that of previous systems \cite{Oh2011} with 0.37 $\mu$V $\pm$ 0.048 $\mu$V noise across all channels.
 thus meeting the criteria outlined in section \ref{motivation}.

Images obtained previously with a 32 channel system by \citet{Aristovich_2016} with mechanical stimulation of the rat whisker which were found to be in good agreement with established literature, and cross validated with other imaging techniques both in initial onset \cite{armstrong1991thalamo}, and spread of activity \cite{petersen2007functional}. Whilst cross validation was not performed in this study, the literature is equally well established regarding forepaw stimulation, to which these images are in good agreement \cite{peeters2001comparing} \cite{masamoto2007relationship} \cite{lowe2007small}.





The system was able to detect voltage changes of the order of 10s of $\mu$V in both stand EIT time difference and stimulation triggered experiments. Physiological variations and noise, as well as differing hardware setups and data analysis techniques make direct comparison to other EIT systems difficult. However, the results from previous EP studies \cite{Oh2011} in the rat cortex have demonstrated a similar noise level to those recorded with the system described here, suggesting physiological noise is dominant. This is also true for the scalp recordings, where the system noise was masked by physiological noise as well as motion artefacts, although not to the same extent as had been found in previous studies \cite{Fabrizi_2006}. 

