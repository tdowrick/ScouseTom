\subsection{System characterisation}
Experiments in a resistor phantom with realistic loads demonstrated an SNR of 77.5 dB $\pm$ 1.3 dB 




when averaged across 100 repeats of a 540 measurement protocol. There were no significant changes in SNR across a four hour recording, or across the frequency range of both EEG systems. 


expanded comparison to other systems. we expect lower SNR based on testing paradigm

cardiff phantom appears noisier
These arise predominantly from temperature changes within the current source, and variations in power supply voltage.



RE as good as KHU and others








The system was able to detect voltage changes of the order of 10s of $\mu$V in both stand EIT time difference and stimulation triggered experiments. Physiological variations and noise, as well as differing hardware setups and data analysis techniques make direct comparison to other EIT systems difficult. However, the results from previous EP studies \cite{Oh2011} in the rat cortex have demonstrated a similar noise level to those recorded with the system described here, suggesting physiological noise is dominant. This is also true for the scalp recordings, where the system noise was masked by physiological noise as well as motion artefacts, although not to the same extent as had been found in previous studies \cite{Fabrizi_2006}. 

